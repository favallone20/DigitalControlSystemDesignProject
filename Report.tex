\documentclass[11pt,a4paper,oneside]{extarticle}
\usepackage[utf8]{inputenc}
\usepackage[italian]{babel}
\usepackage[backend=biber, style=alphabetic, sorting=ynt]{biblatex}
\usepackage{csquotes}
\usepackage{geometry}
\usepackage{amsmath}
\usepackage{hyperref}
\usepackage{graphicx}
\usepackage[labelfont=bf]{caption}
\usepackage{float}

\setlength{\parindent}{0pt}

\newcommand\img[3]{
    \begin{figure}[H]\centering
        \includegraphics[width=#1\textwidth]{imgs/#2.png}
        \caption{#3}
        \label{fig:#2}
    \end{figure}
}


\newcommand\mtrx[1]{
    \begin{bmatrix}
        #1
    \end{bmatrix}
}

\begin{document}

\begin{titlepage}
	\begin{center}

		\Large \textbf{UNIVERSITY OF SALERNO}
		\vspace{0.5cm}

		\normalsize DEPARTMENT OF INFORMATION AND ELECTRICAL
		ENGINEERING AND APPLIED MATHEMATICS
		\vspace{1.5cm}

		\includegraphics[width=0.32\textwidth]{imgs/unisa.png}
		\vspace*{1.5cm}

		\huge \textbf{Report Digital Control Systems Design}
		\vspace*{1cm}

		\Large \textbf{Control of a DC motor with state feedback}
		\vspace*{2cm}

		\textbf{Francesco Avallone} - 0622701488 -
		\href{mailto:f.avallone20@studenti.unisa.it}
		{f.avallone20@studenti.unisa.it}

		\textbf{Lorenzo Pagliara} - 0622701576 -
		\href{mailto:l.pagliara5@studenti.unisa.it}
		{l.pagliara5@studenti.unisa.it}

		\vspace{\fill}
		2021 - 2022
	\end{center}

\end{titlepage}

\tableofcontents

\newpage
\newgeometry{left=2.5cm,bottom=2.5cm, right=2.5cm, top=2.5cm}

\section{Formulation of the control problem formulation at high level}
Nowadays, DC motors are widely used in many different technology 
applications,in particular in the robotics field for the robot's 
joints, but also in the locomotion field and others.
In these contests, it has a significative role the definition of a 
control where it is possible to assign a position or velocity 
reference.\bigskip
In this document has been developed both control approaches: 

\begin{itemize}
	\item position control;
	\item velocity control.
\end{itemize}

For both approaches, it has been used the advanced 
\textit{state feedback control} technique, starting from the state 
space of the mathematical models of the same process on which apply 
the control approaches mentioned before.

\section{Mathematical model}
\subsection{Mathematical model for the position controller}
The DC motor mathematical model in the state space form, which is 
obtained from the DC motor electrical and mechanical equations, is the
following second-order system where the state's variables are the 
angular position $\theta$ and the angular velocity $\omega$:

\begin{equation}
	\begin{split}
		&\mtrx{\dot{\theta}\\ \dot{\omega}}= \mtrx{0 & 1\\ 0 
		& -\dfrac{1}{\tau_m}}\mtrx{\theta\\ \omega} + \mtrx{0\\ 
		\dfrac{k_m}{\tau_m}}v\\ &y = \mtrx{1 & 0}\mtrx{\theta\\ 
		\omega}
	\end{split}
\end{equation}
The control input is the tension applied to the motor $v$ 
which values are beetween $0V$ and $12V$, while the ouput is the
angular position $\theta$ in rad.

\begin{table}[H]
	\centering
	\begin{tabular}{|l|c|l|}
		\hline
		$R$   & $3.2\Omega$       \\ \hline
		$L$   & $8.2mH$           \\ \hline
		$K_e$ & $0.85Vs/rad$      \\ \hline
		$K_t$ & $0.85Nm/A$        \\ \hline
		$b$   & $0.016Nms/rad$    \\ \hline
		$I$   & $0.0059Nms^2/rad$ \\ \hline
	\end{tabular}
	\caption{Motor parameters.}
	\label{tab:motor_parameters}
\end{table}
Considering the parameters in Table \ref{tab:motor_parameters}, it is
possible to obtain the following state space model:

\begin{equation}
	\begin{split}
		&\mtrx{\dot{\theta}\\ \dot{\omega}}= \mtrx{0 & 1\\ 0 & -38.27}\mtrx{\theta\\ \omega} + \mtrx{0\\ 45.02}u\\
		&y = \mtrx{1 & 0}\mtrx{\theta\\ \omega}
	\end{split}
\end{equation}

\subsection{Mathematical model for the velocity controller}
In the same way, applying the Laplace trasformation and multiplying 
for $60/2\pi$, it has been obtained the transfer function of the DC
motor. The control input is the tension $v$ and the ouput is the 
angular velocity $\omega$ in RPM.

\begin{equation}
	G(s) = \frac{\omega(s)}{v(s)} = \frac{9.5493k}
	{(sL + R)(sI + b) + k^2}
\end{equation}
Considering the DC motor parameters in Table 
\ref{tab:motor_parameters}, it is possible to obtain the relative
transfer function:

\begin{equation}
	G(s) = \frac{\omega(s)}{v(s)} = \frac{167773.98}{s^2 + 392.96s + 15992.15}
\end{equation}

\section{Tecniche di controllo}
For both control approaches (position and speed) the state feedback 
technique has been used. The starting point for the application of 
this technique is the obtaining of a representation in the state 
space. In the case of the control in position the starting model 
already met this requirement, while for the speed control it was 
necessary to carry out the transfer function. \bigskip

Having obtained both representations in the state space, the first 
step was to verify the reachability and the observability. Having 
made sure that both of these requirements were verified,and then the 
models were extended with a fictitious state, representing the full error,  
for the rejection of constant disturbances. In order to obtain a law 
of discrete control, such extended representations were then converted 
into models in the discrete time state space. The latter were then 
used to design a feedback state controller, whose gains 
were obtained through LQR.\bigskip

Due to the non-observability of the state, for both approaches, 
a Luenberger observer, also obtained by the LQR technique, was made.

For a more complete view of the design steps followed, please refer 
to the MATLAB code, which is considered an integral part of this 
document.

\begin{center}
	\href{https://github.com/favallone20/DigitalControlSystemDesignProject/blob/master/Speed_Feedback_MATLAB_Simulink/Speed_State_Feedback.mlx}{\textbf{Velocity control}}, \quad
	\href{https://github.com/favallone20/DigitalControlSystemDesignProject/blob/master/Position_State_Feedback_MATLAB_Simulink/Position_State_Feedback.mlx}{\textbf{Position control}} \bigskip
\end{center}
The same control approaches were then implemented using the 
\textit{direct coding}technique, the links, which reference to the source
files, are written below.

\begin{center}
	\href{https://github.com/favallone20/DigitalControlSystemDesignProject/tree/master/Position_State_Feedback_Direct_Coding}{\textbf{Controllo in posizione direct coding}}, \quad
	\href{https://github.com/favallone20/DigitalControlSystemDesignProject/tree/master/Speed_State_Feedback_Direct_Coding}{\textbf{Controllo in velocità direct coding}}
\end{center}

\section{Results obtained and discussion of performance}
Both for the control in position and for the control in speed, it has 
been followed the model based development which previews a 
phase of testing for each step of development. In particular, the 
validation tasks were carried out:

\begin{enumerate}
	\item Model in the Loop (MIL);
	\item Software in the Loop (SIL);
	\item Processor in the Loop (PIL);
	\item Test sul processo reale;
\end{enumerate}
It should be noted that the test phase on the real process must be 
preceded by the Hardware in the Loop (HIL) phase, which was not carried
 out due to the unavailability of the appropriate instrumentation.

\subsection{Results with control in position}
For the simulations of the control in place the following references 
were used: $0$, $-2\pi$, $\pi$, $2\pi$, $0$, updated every $2s$ 
through a Stateflow chart. \bigskip

From the various simulations corresponding to the different validation
tasks it can be noticed that the system response remains almost 
unchanged, with a settling time less than 1 second and a behavior 
quite free of overshoot. The slightly more relevant peaks are 
obtained when the reference changes in form by a value of at least 
$2 \pi$. This behavior depends on the presence of the integral action,
which in the absence of an anti windup scheme, with larger errors 
makes the control input saturated at the limit of the actuator. 
\bigskip

In the figures \ref{fig:Position_State_Feedback_Simulation}, 
\ref{fig:Position_State_Feedback_SIL_Simulation}, 
\ref{fig:Position_State_Feedback_PIL_Simulation} are reported 
the relative outputs  in the MIL, 
SIL, PIL phase. In addition, in the figures  
\ref{fig:Position_State_Feedback_SIL_Code_Profiling} and
\ref{fig:Position_State_Feedback_PIL_Code_Profiling} are reported 
the relative execution time of SIL and PIL phase. 
As can be expected, SIL validation took time 
considerably lower than PIL.\bigskip

Particular attention should be paid to the execution of the control 
algorithm on the physical engine. The responses of the auto-generated 
algorithm and the direct coding algorithm respectively are given in 
Figure \ref{fig:Position_State_Feedback_Motor_Simulation} e in Figure 
\ref{fig:Position_State_Feedback_Direct_Coding}. 
From these figures it is possible to notice that the motor response 
has a slightly oscillatory behavior around the reference value. This 
depends on the physical resolution of the motor that does not allow 
to obtain an angular position perfectly coinciding with the reference.

\subsection{Risultati con controllo in velocità}
The following references were used for speed control simulations: 
$0$, $80$, $125$, $0$, $-80$, $-125$, updated every $2s$ 
through a Stateflow chart. \bigskip

Also in this case in the simulations correspondent the answer of 
the system remains almost unchanged, with times of settling of 
approximately half second and a behavior totally free of overshoot.
\bigskip

In the Figures \ref{fig:Speed_State_Feedback_Simulation}, 
\ref{fig:Speed_State_Feedback_SIL_Simulation}, 
\ref{fig:Speed_State_Feedback_PIL_Simulation} are reported the relative 
motor outputs in the MIL, SIL, PIL validation phases. 
In addition, in the figures 
\ref{fig:Speed_State_Feedback_SIL_Code_Profiling} and
\ref{fig:Speed_State_Feedback_PIL_Code_Profiling} are reported 
the relative execution times of the SIL and PIL validation. 
As can be expected, SIL validation took time considerably lower than 
PIL.\bigskip

Particular attention should be paid to the execution of the control 
algorithm on the physical engine. The responses to the auto-generated 
algorithm and the direct coding algorithm respectively are given in 
Figure \ref{fig:Speed_State_Feedback_Motor_Simulation} and in Figure 
\ref{fig:Speed_State_Feedback_Direct_Coding}. From the figures it can 
be noticed that the response of the motor has of the continuous peaks 
due to the error of the estimate of the speed because of the loss of 
ticks of the encoder.

\newpage
\appendix
\section{Simulations}
\subsection{Position control}

\img{1}{Position_State_Feedback_Simulation}{MIL Simulation with position control}
\img{1}{Position_State_Feedback_SIL_Simulation}{SIL Simulation with position control}
\img{1}{Position_State_Feedback_SIL_Code_Profiling}{Execution times of the SIL simulation with position control}
\img{1}{Position_State_Feedback_PIL_Simulation}{PIL Simulation with position control}
\img{1}{Position_State_Feedback_PIL_Code_Profiling}{Execution times of the PIL simulation with position control}
\img{1}{Position_State_Feedback_Motor_Simulation}{Execution on the real motor of the position control, with the auto-generated code}
\img{1}{Position_State_Feedback_Direct_Coding}{Execution on the real motor of the position control, with the direct coding approach}

\subsection{Velocity control}
\img{1}{Speed_State_Feedback_Simulation}{MIL simulation for velocity control.}
\newpage
\img{1}{Speed_State_Feedback_SIL_Simulation}{SIL simulation with velocity control.}
\img{1}{Speed_State_Feedback_SIL_Code_Profiling}{Execution times of SIL simulation with velocity control}
\img{1}{Speed_State_Feedback_PIL_Simulation}{PIL simulation with control velocity.}
\img{1}{Speed_State_Feedback_PIL_Code_Profiling}{Execution times of PIL simulation with velocity control.}
\img{1}{Speed_State_Feedback_Motor_Simulation}{Execution on the real motor of the velocity control, with the auto-generated code}
\img{1}{Speed_State_Feedback_Direct_Coding}{Execution on the real motor of the velocity control, with the direct coding approach}

\newpage


\end{document}
